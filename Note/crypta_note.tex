\documentclass{llncs}[11pt]
%\usepackage{hyperref}

\newcommand{\tr}[1]{\hbox{}^{T}\negmedspace{#1}}

\usepackage[utf8]{inputenc}  % UTF-8 input encoding

% UNICODE conversion
% greek letters

% Α α Β β Γ γ Δ δ Ε ε Ζ ζ Η η Θ θ Ι ι Κ κ Λ λ Μ μ Ν ν Ξ ξ Ο ο Π π Ρ ρ Σ σ Τ τ Υ υ Φ φ Χ χ Ψ ψ Ω ω

\DeclareUnicodeCharacter{03B1}{\ensuremath{\alpha}}
\DeclareUnicodeCharacter{03B2}{\ensuremath{\beta}}
\DeclareUnicodeCharacter{03B3}{\ensuremath{\gamma}}
\DeclareUnicodeCharacter{0393}{\ensuremath{\Gamma}}
\DeclareUnicodeCharacter{03B4}{\ensuremath{\delta}}
\DeclareUnicodeCharacter{0394}{\ensuremath{\Delta}}
\DeclareUnicodeCharacter{03B5}{\ensuremath{\varepsilon}}
\DeclareUnicodeCharacter{03B6}{\ensuremath{\zeta}}
\DeclareUnicodeCharacter{03B7}{\ensuremath{\eta}}
\DeclareUnicodeCharacter{03B8}{\ensuremath{\vartheta}}
\DeclareUnicodeCharacter{0398}{\ensuremath{\Theta}}
\DeclareUnicodeCharacter{03BA}{\ensuremath{\kappa}}
\DeclareUnicodeCharacter{03BB}{\ensuremath{\lambda}}
\DeclareUnicodeCharacter{039B}{\ensuremath{\Lambda}}
\DeclareUnicodeCharacter{00B5}{\ensuremath{\mu}}      % micron sign
\DeclareUnicodeCharacter{03BC}{\ensuremath{\mu}}
\DeclareUnicodeCharacter{03BD}{\ensuremath{\nu}}
\DeclareUnicodeCharacter{03BE}{\ensuremath{\xi}}
\DeclareUnicodeCharacter{039E}{\ensuremath{\Xi}}
\DeclareUnicodeCharacter{03B9}{\ensuremath{\iota}}
\DeclareUnicodeCharacter{03C0}{\ensuremath{\pi}}
\DeclareUnicodeCharacter{03A0}{\ensuremath{\Pi}}
\DeclareUnicodeCharacter{03C1}{\ensuremath{\rho}}
\DeclareUnicodeCharacter{03C3}{\ensuremath{\sigma}}
\DeclareUnicodeCharacter{03A3}{\ensuremath{\Sigma}}
\DeclareUnicodeCharacter{03C4}{\ensuremath{\tau}}
\DeclareUnicodeCharacter{03C6}{\ensuremath{\varphi}}
\DeclareUnicodeCharacter{03A6}{\ensuremath{\Phi}}
\DeclareUnicodeCharacter{03C7}{\ensuremath{\chi}}
\DeclareUnicodeCharacter{03C8}{\ensuremath{\psi}}
\DeclareUnicodeCharacter{03A8}{\ensuremath{\Psi}}
\DeclareUnicodeCharacter{03C9}{\ensuremath{\omega}}
\DeclareUnicodeCharacter{03A9}{\ensuremath{\Omega}}
\DeclareUnicodeCharacter{03C5}{\ensuremath{\upsilon}}
\DeclareUnicodeCharacter{03A5}{\ensuremath{\Upsilon}}

% some modified characters
\DeclareUnicodeCharacter{1FF6}{\ensuremath{\tilde{\omega}}}

% useful math symbols
\DeclareUnicodeCharacter{221A}{\sqrt}
\DeclareUnicodeCharacter{2264}{\leq}
\DeclareUnicodeCharacter{2265}{\geq}
\DeclareUnicodeCharacter{221E}{\infty}
\DeclareUnicodeCharacter{2211}{\sum}
\DeclareUnicodeCharacter{2208}{\in}
\DeclareUnicodeCharacter{2209}{\notin}
\DeclareUnicodeCharacter{2202}{\partial}
\DeclareUnicodeCharacter{222B}{\int}
\DeclareUnicodeCharacter{2148}{\id}  
\DeclareUnicodeCharacter{2248}{\approx}  
\DeclareUnicodeCharacter{2260}{\neq}  
\DeclareUnicodeCharacter{00B1}{\pm}  
\DeclareUnicodeCharacter{2A02}{\otimes}
\DeclareUnicodeCharacter{2A01}{\oplus}
\DeclareUnicodeCharacter{00BD}{\nicefrac{1}{2}}
\DeclareUnicodeCharacter{00D7}{\times}
\DeclareUnicodeCharacter{00B7}{\cdot}
\DeclareUnicodeCharacter{22C5}{\cdot}
\DeclareUnicodeCharacter{2190}{\gets}

% the prime
\DeclareUnicodeCharacter{2032}{\ensuremath{'}}

% daggers don't need to be preceded by a '^' anymore
\DeclareUnicodeCharacter{2020}{^{\dagger}}
\DeclareUnicodeCharacter{1D40}{^{\textnormal{\tiny{T}}}}
\DeclareUnicodeCharacter{00B2}{^{2}}
\DeclareUnicodeCharacter{00B3}{^{3}}
\DeclareUnicodeCharacter{00BD}{\ensuremath{\frac{1}{2}}\xspace}

% bracket heaven
\DeclareUnicodeCharacter{2329}{\langle}
\DeclareUnicodeCharacter{232A}{\rangle}
\DeclareUnicodeCharacter{27E8}{\langle}
\DeclareUnicodeCharacter{27E9}{\rangle}
\DeclareUnicodeCharacter{2016}{\|}

% fun with arrows
\DeclareUnicodeCharacter{2192}{\to}
\DeclareUnicodeCharacter{21D2}{\implies}
\DeclareUnicodeCharacter{21A6}{\mapsto}

% specialized math letters
\DeclareUnicodeCharacter{2113}{\ell}
\DeclareUnicodeCharacter{210B}{\h}
\DeclareUnicodeCharacter{2115}{\bN}
\DeclareUnicodeCharacter{211D}{\bR}
\DeclareUnicodeCharacter{2102}{\bC}
\DeclareUnicodeCharacter{2119}{\Pr}
\DeclareUnicodeCharacter{2130}{\sE}
\DeclareUnicodeCharacter{2133}{\cM}

% other notation
\DeclareUnicodeCharacter{00B7}{\ensuremath{\cdot}\xspace}
\DeclareUnicodeCharacter{2261}{\ensuremath{\equiv}\xspace}


\usepackage{graphicx}
\usepackage{verbatim}
\usepackage{amsmath,amsfonts,amssymb}
\usepackage{booktabs}
\usepackage[linesnumbered,vlined]{algorithm2e}

\usepackage{graphics}
\usepackage{tikz}
\usepackage{pgfplots}
%\pgfplotsset{compat=1.11}

\usepackage[english]{babel}
\usepackage{todonotes}

\newtheorem{exmp}{Example}
\newtheorem{assumption}{Assumption}
\newtheorem{result}{Result}

\usepackage{a4wide}
\setlength\parindent{0pt}

\usepackage{cite}
\usepackage{url}
\usepackage{times}
\newcommand{\F}{\ensuremath{\mathbb{F}}}


\newcommand{\heading}[1]{{\vspace{6pt}\noindent\sc{#1.}}}
\newcommand{\Dreg}{D_{\mathrm{reg}}}

\newcommand*{\neper}{e}

\newcommand{\GAPSVP}[1]{\ensuremath{\textsc{GapSVP}_{#1}}\xspace}

\newcommand{\AroraGe}{Arora \& Ge\xspace}
\newcommand{\prooffactor}{\ensuremath{\sigma \, q \log q}\xspace}

\newcommand{\bigO}[1]{\ensuremath{\mathcal{O}\left({#1}\right)}\xspace}
\newcommand{\tildeO}[1]{\ensuremath{\tilde{\mathcal{O}}(#1)}\xspace}
\newcommand{\poly}{ {\rm poly}(n)}
\newcommand{\chig}{\ensuremath{\chi_{\alpha,q}}}
\newcommand{\U}{\ensuremath{\mathcal{U}\xspace}}
\newcommand{\Z}{\ensuremath{\mathbb{Z}}}
\newcommand{\Zq}{\ensuremath{\mathbb{Z}_q}}
\newcommand{\Ldis}{L_{\mathbf{s},\chi}^{(n)}}
\newcommand{\bLdis}{L_{\mathbf{s},\mathcal{U}(\mathbb{F}_2)}^{(n)}}
\newcommand{\TLdis}{L_{\mathbf{s},\mathcal{U}([-T\ldots,T])}^{(n)}}

\newcommand{\Bdis}[1]{B_{\mathbf{s},\chi,#1}^{(n)}}
\newcommand{\sample}{\ensuremath{\leftarrow_{\$}}}

\newcommand{\CDF}{\ensuremath{\textnormal{CDF}}}
\newcommand{\E}{\ensuremath{\textnormal{E}}}
\newcommand{\Var}{\ensuremath{\textnormal{Var}}}
\newcommand{\Sample}{\ensuremath{\mathbf{Sample}}\xspace}

\newcommand{\abs}[1]{\ensuremath{|#1|}\xspace}

\newcommand{\avec}{\ensuremath{\mathbf{a}}\xspace}
\newcommand{\cvec}{\ensuremath{\mathbf{c}}\xspace}
\newcommand{\evec}{\ensuremath{\mathbf{e}}\xspace}
\newcommand{\svec}{\ensuremath{\mathbf{s}}\xspace}
\newcommand{\tvec}{\ensuremath{\mathbf{t}}\xspace}
\newcommand{\vvec}{\ensuremath{\mathbf{v}}\xspace}

\newcommand{\sol}{\ensuremath{(s_{n-w},\dots,s_{n-1})}\xspace}
\newcommand{\solvec}{\ensuremath{\mathbf{s'}}\xspace}

\newcommand{\dotp}[2]{\ensuremath{\langle {#1},{#2}\rangle}\xspace}
\newcommand{\N}[1]{\ensuremath{\mathcal{N}({#1)}}}
\def\abn{\lceil n/b \rceil}
\def\gnd{\lceil n/d \rceil}

\DeclareMathOperator{\erf}{erf}

\newcommand{\lowerbound}{\ensuremath{\lceil -q/2 \rceil}\xspace}
\newcommand{\upperbound}{\ensuremath{\lfloor q/2\rfloor}\xspace}


%Intial AG
\newcommand{\id}{\mathcal{I}}

\newcommand{\Mac}[1]{\ensuremath{\mathcal{M}^{\mathrm{acaulay}}_{#1}}}
\newcommand{\K}{\ensuremath{\mathbb{K}}}


%\newcommand{\Zq}{\ensuremath{\mathbb{Z}_q}}
\newcommand{\Zb}{\ensuremath{\mathbb{Z}_2}}
\newcommand{\Rn}{\ensuremath{{\bf R}_n[X]}}
\newcommand{\Ldisring}{Lr_{\mathbf{s},\chi}^{(n)}}

\newcommand{\ZqX}{\ensuremath{\mathbb{Z}_q}[x_1,\ldots,x_n]}
\newcommand{\ZqXh}{\ensuremath{\mathbb{Z}_q}^{(d)}[x_1,\ldots,x_n]}
\newcommand{\ZbX}{\ensuremath{\mathbb{Z}_2}[x_1,\ldots,x_n]}
\newcommand{\KX}{\ensuremath{\mathbb{K}}[x_1,\ldots,x_n]}

\newcommand{\red}[1]{\textcolor{red}{[L.P. :#1]}}

\newcommand{\Sdis}{\ensuremath{S^{(m,n,d)}_{\mathbf{s}}}}
\newcommand{\pos}{PoSSo}

\newcommand{\nbvar}{n}
\newcommand{\nbeq}{m}

%\newcommand{\Udis}{\mathcal{U}^{(m,n,d)}_{{\rm PoSSo}}}
%\newcommand{\Udispwe}{\mathcal{U}^{(m,n,d)}_{{\rm PWE}}}
%sff

\newcommand{\Udis}{\mathcal{U}^{(m,n,d)}}
\newcommand{\Udispwe}{\mathcal{U}^{(m,n,d)}}

\newcommand{\Pdis}{P^{(m,n,d)}_{\mathbf{s},\chi}}
\newcommand{\Pdisu}{P^{(n,d)}_{\mathbf{s},\chi}}

\newcommand{\Pdish}{H^{(m,n,d)}_{\mathbf{s},\chi} }

\newcommand{\Adv}{{\bf Adv}}
\newcommand{\s}{{\mathbf{s}}}

\renewcommand{\Pr}{\textnormal{Pr}}
\def\rand{\stackrel{{}_{\$}}{\leftarrow}} 


%Names 
\newcommand\LWE{\ensuremath{{\rm LWE}}\xspace}

%Section on GB 


\def\DAG{D_{{\rm AG}}}
\def\MAG{M_{{\rm AG}}}
\def\CAG{C_{{\rm AG}}}

\def\DGB{D_{{\rm GB}}}
\def\MGB{M_{{\rm GB}}}
\def\CGB{C_{{\rm GB}}}


%%% Local Variables:
%%% mode: plain-tex
%%% TeX-master: "structured_noise"
%%% End:

\usepackage{graphics}

\usepackage[index,multiuser]{fixme} 
\fxsetup{
   status=draft,
   layout=marginnote, % also try footnote or pdfnote
   theme=color
}
\FXRegisterAuthor{malb}{amalb}{Martin}
\FXRegisterAuthor{lp}{lpe}{LP}
\FXRegisterAuthor{rf}{arf}{Rob}

\parindent 0pt
\parskip 3pt plus 1pt minus 1pt
\usepackage{boxedminipage}



\begin{document}

\pagestyle{headings}{\mainmatter}


\title{CRTYPA}

\author{Ludovic Perret}
\institute{ 
Sorbonne Universit\'es, UPMC Univ Paris 06, INRIA Paris \\ 
LIP6, {\tt PolSyS} Project, Paris, France 
}


%\author{Martin R.~Albrecht \inst{1} \and Carlos Cid \inst{1} \and Jean-Charles Faug\`ere \inst{2} \and Ludovic Perret \inst{2}}
%\institute{
%Information Security Group\\
%Royal Holloway, University of London\\
%Egham, Surrey TW20 0EX, United Kingdom \and
%INRIA, Paris-Rocquencourt Center, POLSYS Project\\
%UPMC Univ Paris 06, UMR 7606, LIP6, F-75005, Paris, France\\
%CNRS, UMR 7606, LIP6, F-75005, Paris, France\\
%\email{martin.albrecht@rhul.ac.uk, carlos.cid@rhul.ac.uk, jean-charles.faugere@inria.fr, ludovic.perret@lip6.fr}  
%}


\maketitle

\begin{abstract}

\end{abstract}

\section{Rappels}

\subsection{Chiffrement  à clé secrète}
On rappelle ici quelques outils permettant de garantir la {\bf confidentialité} d'une donnée. 
\subsection{Chiffrement  par flot}
\begin{definition}
Un {\bf chiffrement par flot} est un chiffrement à clef secrète. Il est constitué d'un premier algorithme ${\rm SC} : \F_2^t \to  \{0,1\}^n$ qui prend en entrée une clé secrète $K \in \F_2^k$.  La fonction ${\rm SC}_K$ permet de générer une {\bf  suite chiffrante} à partir de la clef secrète $K$.  
On chiffre alors une message $m \in \{0,1\}^n$ par:   
$$
c={\rm SC}(K)  \oplus m.      
$$ 
Pour déchiffrer, on calcule: 
$$
m={\rm SC}(K)  \oplus c.      
$$ 
\end{definition}

\begin{example}
${\tt RC}4$  est un exemple célèbre de chiffrement par flot. {\bf Il ne faut surtout pas utiliser ${\tt RC}4$ en pratique.} 

\end{example}
\subsection{Chiffrement  par bloc}


\begin{definition}
Un {\bf chiffrement par bloc} est un chiffrement à clef secrète. C'est 
une fonction $E_K : \F_2^n \to  \F_2^n$, paramétrée par une clef secrète $K \in \F_2^k$, qui opère sur un bloc de taille fixe. On associe à $E_K$ 
une fonction de déchiffrement  $D_K : \F_2^n \to  \F_2^n$ telle que:
$$
D_K\big( E_K(m) \big)=m, \quad \forall m \in \F_2^n. 
$$    
Ainsi, pour chiffrer un message $m \in \F_2^n$, on calcule:
$$
c=E_K(m) \in \F_2^n. 
$$ 
Pour déchiffrer  $c \in \F_2^n$, on fait:
$$
m=D_K(m) \in \F_2^n. 
$$
\end{definition}
\begin{example}
En chiffrement par bloc, le standard est {\tt AES}128 dans lequel 
$n=128$ (taille du bloc) et $t=128$ (taille de la clé secrète).   
\end{example}

\begin{definition}
Un {\bf mode opératoire} pour un chiffrement par bloc est une algorithme dont l'objectif est de chiffrer un message de taille quelconque $m \in \{0,1\}^*$   
avec un chiffrement par bloc  $E_K : \F_2^n \to  \F_2^n$. Le principe est de découper le message en des blocs de taille $n$ et d'utiliser $E_K$ sur chaque bloc.     
\end{definition}
\begin{example}
Soit $m=(m_1,\ldots,m_t) \in  (\F_2^n)^t$. 
\begin{itemize}
\item Un mode opératoire simple est le mode {\tt ECB} qui consiste à chiffrer chaque bloc du message $m$ indépendamment. C'est à dire, on calcule:
$$
c_i=E_K(m_i), \forall i, 1 \leq i \leq t. 
$$      
On déchiffre par:
$$
m_i=D_K(c_i), \forall i, 1 \leq i \leq t. 
$$      
{\bf Il ne faut surtout pas utiliser {\tt ECB} en pratique.}
\item Le  mode {\tt CBC} fonctionne de la manière suivante. Nous avons un vecteur public d'initialisation $c_0={\rm IV} \in \F_2^n$. On chiffre comme:
$$
c_i=E_K(m_i \oplus c_{i-1}), \forall i, 1 \leq i \leq t. 
$$  
Pour le déchiffrement, nous avons:
$$
m_i=D_K(c_i) \oplus c_{i-1}, \forall i, 1 \leq i \leq t. 
$$  
 \item Le  mode {\tt CTR} fonctionne de la manière suivante. Nous avons un vecteur public d'initialisation ${\rm IV} \in \F_2^n$. On chiffre comme:
$$
c_i=m_i \oplus E_K( {\rm IV}  \oplus i), \forall i, 1 \leq i \leq t. 
$$  
%Pour le déchiffrement, nous avons:
%$$
%m_i=D_K(c_i) \oplus c_{i-1}, \forall i, 1 \leq i \leq t. 
%$$  
 
\end{itemize}

\end{example}
\subsection{Hachage}
\begin{definition}
Une {\bf fonction de hachage} est une fonction qui prend comme entrée une donnée de taille quelconque et retourne une {\bf emprunte} de taille fixe.
Autrement dit, une fonction de hachage est une fonction de la forme  
$H : \{0,1\}^* \to  \F_2^n$.
Une {\bf fonction de hachage cryptographique} est une fonction de hachage
$H : \{0,1\}^* \to  \F_2^n$ telle que:
\begin{itemize}
\item $H$ est facilement évaluable, i.e. $\forall D \in  \{0,1\}^*$, $ H(D)$ est calculable en temps polynomial.
\item $H$ est {\bf résistante à la pré-image}, i.e. $\forall h \in \F_2^n$, il est difficile  de trouver $D \in  \{0,1\}^*$ tel que  $ H(D)=h$.  
\item $H$ est {\bf résistante à la seconde  pré-image}, i.e. $\forall D \in \{0,1\}^*$ fixé, il est difficile de trouver $D' \in  \{0,1\}^*$ tels que 
$$
H(D)=H(D') \mbox{ et }  D \not =D'.
$$  
\item $H$ est {\bf résistante à la collision}, i.e. il est difficile de trouver un 
couple $(D,D') \in \{0,1\}^*$ tels que 
$$
H(D)=H(D') \mbox{ et }  D \not =D'.
$$  
\end{itemize}

\end{definition}

\begin{remark}
Soit  $H : \{0,1\}^* \to  \F_2^n$ une fonction de hachage. Le {\bf paradoxe des anniversaires} permet de trouver une collision en $O(2^{n/2})$ évaluations de $H$ avec une forte probabilité.     
\end{remark}

\begin{remark}
\begin{itemize}
\item ${\tt MD}5$ fonction de hachage dans laquelle $n=128$. {\bf Il ne faut surtout pas utiliser ${\tt MD}5$ en pratique.} On trouve, par exemple, des collisions dans ${\tt MD}5$ (quasiment) en temps réel.  
\item ${\tt SHA}1$ fonction de hachage dans laquelle $n=160$. {\bf Il ne faut surtout pas utiliser ${\tt SHA}1$ en pratique.} Google, en collaboration avec de nombreux chercheurs, a annoncé le calcul d'une collision pour ${\tt SHA}1$. Cette collisions a nécessité de l'ordre de $2^{63.1}$ évaluations de ${\tt SHA}1$.
\item On peut utiliser  les fonctions de hachage la famille ${\tt SHA}2$ (${\tt SHA}256,{\tt SHA}384$, ou ${\tt SHA}512$)
\item Le nouveau standard est  ${\tt SHA}3$.
\end{itemize}
\end{remark}

\subsection{Authentification à clé secrète}


\begin{definition}
Un {\bf Message Authentication Code (MAC)} est une fonction  ${\rm MAC}_K : \{0,1\}^* \to  \F_2^n$ qui est paramétrée par une clef secrète $K \in \F_2^k$.
Elle prend en entrée une donnée de taille quelconque et retourne un authentifiant de taille fixe. Dans ce modèle, l'émetteur et le destinataire  partagent une clef secrète $K \in \F_2^t$. Ainsi, on associe à une donnée $D \in  \{0,1\}^*$ un authentifiant  $T=\F_2^n$. L'authentifiant $T$ est ainsi envoyé avec la donnée $D$.         

\end{definition}

%\newpage

%\section{Justifications of our Assumptions}\label{sec:assumptions}
%\input{assumptions}


\bibliographystyle{plain}
\bibliography{abbrev3,crypto_crossref,local,jacm,issac}

\appendix



\end{document}

%%% Local Variables:
%%% mode: latex
%%% TeX-master: t
%%% End:
